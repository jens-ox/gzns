\section*{Neurologische Krankheitsbilder}

\paragraph{Strategie}
\begin{itemize}
  \item \textbf{Anamnese}: Erhebung Krankheitsgeschichte
  \item \textbf{Untersuchung}: klinisch + neurologisch
  \item \textbf{Adaptive Diagnostik}
  \item \textbf{Therapie} + Planung
  \item \textbf{Nachuntersuchungen} (\emph{follow-up})
\end{itemize}

\paragraph{Übersicht neurologische Erkrankungen}

Unterscheidung \textbf{zerebral} (Großhirn betreffend) + \textbf{spinal} (zur Wirbelsäule, zum Rückenmark gehörend)
\begin{itemize}
  \item \textbf{Zirkulationsstörungen}
  \begin{itemize}
    \item Infarkt: Gewebsuntergang (Nekrose) infolge einer Sauerstoffunterversorgung (Hypoxie) durch unzureichenden Blutzufluss (Ischämie)
    \item Infarzierung: Hypoxie durch Abluffhindernis
    \item Thrombose: Bildung Blutgerinnsel (Thrombus) in Blutgefäß
    \item Blutung (Hämorraghie): Austreten von Blut aus beliebigem Bereich von Blutbahn/-kreislauf
  \end{itemize}
  \item \textbf{Raumforderungen}: Benigne + maligne Tumore, Blutungen
  \item \textbf{Infektionen}: Bakteriell, Viral, durch Prionen (Proteine, die in normalen \emph{und} pathogenen Konformationen vorliegen können), Parasitär (Protozonen (Urtiere), Würmer, Pilze), Autoimmunerkrankungen (zB multiple Sklerose)
  \item \textbf{Gefäßmissbildungen}:
  \begin{itemize}
    \item Aneurysmen: Krankhafte Gefäßaussackung
    \item Angiome: Tumorartige Gefäßneubildung oder Gefäßfehlbildung
    \item Kavernome: Hämangiom (= Gefäßmissbildung)
  \end{itemize}
  \item \textbf{Epilepsien}
  \item \textbf{Stoffwechselerkrankungen}: Speicherkrankheiten, Vergiftungen, \dots
  \item \textbf{Erkrankungen PNS}
  \item \textbf{Myopathien} (Muskelerkrankungen)
  \item \textbf{Neuropathien}
  \item \textbf{Systemerkrankungen ZNS}:
  \begin{itemize}
    \item Nukleäre Atrophien: Gewebe-Verkleinerungen
    \item Spinalparalyse
    \item Amyotrophe Lateralsklerose (ALS)
  \end{itemize}
  \item \textbf{Psychiatrische Störungen}
  \begin{itemize}
    \item Schizophrenie, Psychose, Neurose, Angst
    \item Anpassungsstörungen, Persönlichkeitsstörungen
  \end{itemize}
\end{itemize}

\paragraph{Querschnittssyndrom}
\begin{itemize}
  \item[=] Schädigung Rückenmark in gesamtem Durchmesser
  \item[\( \to \)] darunterliegende Spinalnerven funktionslos 
\end{itemize}

\paragraph{Epilepsie}
\begin{itemize}
  \item \textbf{Krankheitsbild}: spontan auftretende Krampfanfälle
  \item \textbf{Grund}: Folge anfallsartiger (paroxysmaler) synchroner Entladungen von Neuronengruppen im Gehirn \( \to \) plötzliche, unwillkürliche Verhaltens- oder Befindensstörungen
  \item \textbf{Diagnostik}: Anamnese, Hirnstromkurve mit EEG, ggf Hirn-MRT
  \item \textbf{Therapie}: Antikonvulsiva (krampfunterdrückende Medikamente)
  \begin{itemize}
    \item therapieresistent \( \to \) operative Methoden
  \end{itemize}
  \item \textbf{Gelegenheitskrämpfe} (Fieberkrämpfe): Haben 4-5\% aller Menschen wenige Male im Leben
  \item \textbf{Fokale Anfälle}: nur eine Hirnregion in einer Gehirnhälfte betroffen
  \begin{itemize}
    \item[\( \to \)] Erkrankung geht von ``Krankheitsherd'' aus (Ausgangsstelle)
  \end{itemize}
  \item \textbf{Aura}: Sinneswahrnehmungen, die bei manchen Menschen kurz danach eintreffenden epileptischen Anfall ankündigen
  \item \textbf{Generalisierte Anfälle}: Betrifft gesamtes Hirn, Anfallsverlauf und Symptome zeigen keine anatomisch begrenzte Lokalisation
  \item \textbf{Absencen}: Bewusstseinspausen (\emph{petit-mal})
  \item \textbf{Tonisch-klonische Anfälle}: Anfall mit Bewusstseinsverlust, Sturz, Verkrampfung, rhythmische Zuckungen, Zungenbiss (\emph{grand-mal})
  \item \textbf{Astatische Anfälle}: Sturzanfälle mit atoner Muskulatur (Erschlaffung)
  \item \textbf{Status epilepticus}: Anfallsserie (ggf lebensbedrohlich)
\end{itemize}

\paragraph{Gehirntumore --- Begriffe}
\begin{itemize}
  \item \textbf{Ossär}: Knochen betreffend
  \item \textbf{Neurogen}: Nervensystem betreffend
  \item \textbf{Vaskulär}: Blutgefäße betreffend
  \item \textbf{Intrasellär}: Liegt in Sella (Teil von Schädelknochen), beinhaltet Drüse (Hirn\-an\-hangs\-drüse), die wichtige Botenstoffe herstellt
  \item \textbf{Rezidiv}: Wiederauftreten einer Krankheit
\end{itemize}

\paragraph{Gehirntumore --- Gutartig}
\begin{itemize}
  \item \textbf{Definition}: Basierend auf Morphologie/Histologie, Wachstumsverhalten (Zellteilungsindex, Verteilung in andere Gewebe (\emph{infiltrativ})), Lokalisation, Größe, ärztliche Einschätzung, klinische und statistische Ergebnisse
  \item \textbf{Hirneigene Tumore}:
  \begin{itemize}
    \item Astrozytome: ursprünglich in Astrozyten (gehören zu ZNS-Stützgewebe)
    \item Gliom: Sammelbegriff für einige ZNS-Hirntumoren, entstehen aus Gliazellen (Stütz- und Nährgewebe von Nervenzellen), treten meist in Gehirn auf, aber auch in Rückenmark und Hirnnerven möglich
    \item Oligodendrogliome: Neuroepithelialer Tumor, geht vermutlich von Oligodendrozyten (Glia-Telltyp) aus
    \item Mischgliome/Oligoastrozytome: diffuse Gliome, weisen Anteile von Oligodendrogliom und Astrozytom auf
    \item Ependymome
  \end{itemize}
  \item Meningeome, Hypophysenadenome, Neurinome, Hämangioblastome, Dysplastome
  \item \textbf{Kraniopharyngeom}: Ensteht durch Fehlbildung von Restgewebe bei Hirnanhangsdrüse
\end{itemize}

\paragraph{Fistel}
\begin{itemize}
  \item[=] pathologische oder künstlich angelegte rohrförmige Verbindung zwischen zwei Hohlorganen oder zwischen Organ und Körperoberfläche
  \item \textbf{Koagulation}: Gerinnung durch Wärmeentwicklung
  \item \textbf{Dura}: Äußerste Haut
\end{itemize}

\paragraph{Anatomische Hauptrichtungen}
\begin{itemize}
  \item dorsal: rückenseits, am Rücken gelegen
  \item ventral: bauchseits, am Bauch gelegen
  \item kranial: zum Schädel hin
  \item kaudal: zum Schwanz (Gesäß) hin
  \item proximal: zum Körperzentrum hin
  \item distal: vom Körperzentrum entfernt
  \item medial: in der Mitte gelegen
  \item lateral: seitlich
\end{itemize}

\paragraph{Aneurysma}
\begin{itemize}
  \item \textbf{Ursachen} (Pathologie): Embryonaler Gefäßwanddefekt, Gefäß\-teilungs\-stel\-len, Exo\-gene Faktoren, Entzündungen, Hämodynamik
  \item \textbf{Ruptur}: Riss Gefäßwand durch Schwäche des Gefäßes, je nach Gefäßlage lebensbedrohlich
  \item \textbf{Subarachnoidalblutung} (SAB): Freies Blut gelagt in mit Hirnflüssigkeit (\emph{liquor cerebrospinalis}) gefüllten Subarachnoidalraum
  \item \textbf{Symptome innozenter Aneurysmen}: Symptomlos, Warnblutungen, anfallartige Kopfschmerzen, Visusminderung (Sehschärfe), Schädigung von Nerven
  \item \textbf{Symptome ruptierter Aneurysmen}: plötzlicher vernichtender Kopfschmerz, Bewusstseinstrübung, Bewusstlosigkeit, Nackenschmerz, Nackensteifheit, Steigung intrakranieller Druck (ICP), Abfall zerebraler Perfusionsdruck
\end{itemize}