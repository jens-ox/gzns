\section*{Neurodiagnostik}

\paragraph{Diagnostik}
Unterscheidung invasive und nicht-invasive Diagnostik
\begin{itemize}
  \item \textbf{Lumbalpunktion} (Liquordiagnostik): Hohlnadel wird in Lumbalkanal auf Lendenhöhe eingeführt + Nervenwasser entnommen
  \item \textbf{Ultraschall}
  \item \textbf{Elektrophysiologie}
  \item \textbf{Positronen-Emmisions-Tomographie} (PET, Radiologische Diagnostik):
  \begin{enumerate}
    \item Zerfall eines Radionuklids \( \to \) Positron entsteht
    \item Positron trifft auf Elektron \( \to \) Annihilation
    \item Zwei Protonen entstehen (Gammastrahlung) \( \to \) Abstrahlung
    \item Winkel 180° zwischen Gammastrahlen \( \to \) treffen je auf Detektor
  \end{enumerate}
\end{itemize}

\paragraph{Elektrophysiologie}
\begin{itemize}
  \item[=] Neurophysiologie-Teilbereich, befasst sich mit elektrochemischer Signalübertragung in Nervensystem
  \item \textbf{Klinische Elektrophysiologie}: Neurologie-Teilbereich, Unterschiedliche Methoden zur Messung ganzer polysynaptischer zentraler Nervenbahnen + peripherer Nerven
  \begin{itemize}
    \item Methoden: Elektroenzephalographie (EEG), Messung evozierter Potenziale (somatosensorisch, motorisch, visuell, akustisch evoziert), Elektroneurographie (ENG) mit Messung Nervenleitgeschwindigkeit (NLG), Elektromyographie (EMG)
  \end{itemize} 
\end{itemize}

\paragraph{Elektroenzephalographie (EEG)}
\begin{itemize}
  \item Wegen CT + MRT nicht mehr häufig eingesetzt
  \item Aufzeichnung Hirnströme als Maßeinheit elektrischer Hirnaktivität (5-100 \( \mu \)V)
  \item \textbf{Oberflächen-EEG}:\@ Messung mit auf Kopfhaut aufgebrachten Elektroden
  \item Spannungsunterschiede zwischen Elektroden (= Hirnströme) auf Monitor wellenförmig sichtbar
  \item Einteilung Hirnströme nach Frequenz in 3-4 Rhythmen
  \item \textbf{Frequenz schnell} \( \to \) Person wach, Hirnaktivität normal
  \item \textbf{Frequenz langsam} \( \to \) Schlafstadium oder krankhafter Befund
  \item \textbf{Alpha-Wellen} (8-13 Hz): wach, entspannt
  \item \textbf{Beta-Wellen} (14-30 Hz): Medikamente oder fehlende Entspannung
  \item \textbf{Gamma-Wellen} (>30 Hz): Starke Konzentration
  \item \textbf{Theta-Wellen} (4-7 Hz): bei Kindern/Jugendlichen normal
  \item \textbf{Delta-/Subdelta-Wellen} (0.5-3 Hz): Tiefschlaf, Trance
  \item \textbf{Spiked/Sharp Waves}: Epilepsie
\end{itemize}

\paragraph{Nervenleitgeschwindigkeit (NLG)}
\begin{itemize}
  \item \textbf{Prinzip}:
  \begin{enumerate}
    \item Kurzer elektrischer Impuls am Arm oder Beim \( \to \) Reizung Nerv
    \item Nerv depolarisiert \( \to \) Weiterleitung in beide Richtungen
    \item[\( \to \)] Messung ausgelöste Spannungsänderung entlang Nerv
  \end{enumerate}
  \item \textbf{Berechnung}:
  \begin{itemize}
    \item Nerv an zwei Orten stimulieren
    \item Reizantworten im Muskel messen
    \item Differenz Leitungszeiten (Latenz, ms) und Reizorte (mm) bestimmen
    \item NLG = \( \Delta \text{mm} / \Delta \text{ms} \)
  \end{itemize}
  \item \textbf{Verwendung}: Ort + Schwere von Nervenschaden ermitteln, z.B. Polyneuropathie oder Nervenkompressionssyndrom (Karpaltunnelsyndrom)
\end{itemize}

\paragraph{Elektromyographie (EMG)}
\begin{itemize}
  \item[=] elektrische Muskelaktivität messen durch Einstechen von dünner Nadelelektrode in Muskel \( \to \) Ableitung von Potentialschwankungen einzelner motorischer Einheiten durch konzentrische Nadelelektroden
  \item Feststellbar, ob Muskel- oder Nerv-Erkrankung bei Muskelschwäche \\* \( \to \) Differenzierung zwischen Myo- und Neuropathien 
  \item \textbf{Intraoperativ}: EMG als Monitoring von Rückenmarksfunktion bei Wirbelsäulenoperation oder Registrierung Hirnnervenfunktionen bei Hirnstamm-Operationen
\end{itemize}

\paragraph{Evozierte Potentiale}
\begin{itemize}
  \item \textbf{Prinzip}:
  \begin{enumerate}
    \item Sinnesreiz \( \to \) el. Potentialänderung in sensorischen Großhirnrinde-Arealen
    \item Wesentlich kleinere Amplituden (schwer erfassbar)
    \item[\( \to \)] Evozierte Aktivität = \( \varnothing \) mehrere evozierte Potentiale
  \end{enumerate}
  \item \textbf{Somatosensorisch evozierte Potentiale} (SSEP): Beurteilung zentrale somatosensible Leitungsbahnen + peripherer, sensibler Nerven
  \item \textbf{Visuell evozierte Potentiale} (VEP): Beurteilung Sehnerv und -bahn
\end{itemize}

\paragraph{Sonographie (Ultraschall)}
\begin{itemize}
  \item \textbf{Ultraschall}: Schall oberhalb des hörbaren Frequenzbereichs (20 kHz - 1GHz)
  \item \textbf{Sonographie} (Echographie): Ultraschall als bildgebendes Verfahren für medizinische Untersuchung organisches Gewebe
  \item \textbf{Ultraschallgerät}: Elektronik Schallerzeugung, Signalverarbeitung + -dar\-stel\-lung, Schnittstellen für Monitor/Drucker/Speichermedien, auswechelbare Ultraschallsonde (Schallkopf)
  \item \textbf{Schallkopf}:
  \begin{itemize}
    \item Kristalle, die bei Wechselspannung mitschwingen (piezoelektrischer Effekt)
    \item Sendet Schwingungen \( \to \) unterschiedliche Reflektion d. Organe/Gewebe
    \item Impedanz: Wellen-Ausbreitung entgegenwirkender Widerstand
    \item Grauwert = Reflexionsstärke (hohe Reflexion an Grenzflächen zweier Stoffe mit großem Impedanzunterschied)
  \end{itemize}
  \item \textbf{Dopplereffekt}: Bestimmung Blutflussgeschwindigkeit \( \to \) reflektiertes Signal um Frequenz relativ zu ausgesandter verschoben (Dopplersonographie)
\end{itemize}

\paragraph{Röntgen}
\begin{itemize}
  \item \textbf{Röntgendiagnostik}: Körper mit kurzwelliger, unsichtbarer Strahlung durchstrahlen (Wellenlänge 0.01 - 10nm)
  \item \textbf{Durchleuchtung}: Durchstrahltes Gewebe schwächt Strahlung ab \( \to \) Darstellung mit fluoreszierendem Schirm/Bildverstärker
  \item \textbf{Radiographie}: Sichtbar-machen auf Filmmaterial oder durch elektronische Sensoren (digitale Radiographie)
  \item \textbf{Erzeugung}: Elektronen von Glühwendel (Kathode) beschleunigt, treffen auf Anode \( \to \) Abbremsen, entstehung von Bremsstrahlung (= Röntgenstrahlung) + viel Wärme
  \item Röntgenstrahlen-Absorption durch Gewebe dichteabhängig \\* \( \to \) keine Abbildung des Körperinneren möglich
  \item Häufigste Indikation bei Verdacht auf Knochenbruch
  \item Unterschiedliche Strahlenqualitäten (weich/hart), um unterschiedlich dichte Gewebe (Fett, Muskel, Knochen) zu durchdringen
  \begin{itemize}
    \item wenige kV auf Röntgenröhre \( \leadsto \) weiche Strahlung
    \item 25-35kV (Mammographie), 38-120kV (Rest)
    \item weicher \( \to \) höhere Absorption \( \to \) höhere Strahlenbelastung
    \item weicher \( \to \) feinere Gewebeunterschiede sichtbar
    \item härter \( \to \) durchdringt mehr (>100kV durchdringt sogar Bleischürzen)
    \item härter \( \to \) weniger Kontrastunterschiede
  \end{itemize}
\end{itemize}

\paragraph{Natürliche Strahlenbelastung}
\begin{itemize}
  \item \textbf{Maßeinheit}: Millisievert (mSv)
  \item \textbf{Äquivalentdosis}: Dosisgröße für ionisierende Strahlung
  \begin{itemize}
    \item berücksichtigt übertragene Energiedosis
    \item berücksichtigt relative biologische Wirksamkeit (RBW) von Strahlenarten
  \end{itemize}
  \item \textbf{Kosmische Strahlung}: 0.3mSv/Jahr. Entsteht in äußerer Atmosphäre durch Kollision von Wasserstoff-Atomkernen und Luftmolekülen
  \item \textbf{Terrestrische Strahlung}: 0.4mSv/Jahr. Emittiert durch Radionuklide in Böden/Gesteinen der Erdkruste
  \item \textbf{Innere Strahlung}: 1.4mSv/Jahr. Emittiert durch Zerfall natürlicher radioaktiver Stoffe, Aufnahme durch Essen, Trinken, Atmen
  \item \textbf{Hauptbelastung}: Inhalation von Radon
\end{itemize}

\paragraph{Künstliche Strahlenbelastung}
\begin{itemize}
  \item \textbf{Medizinische Anwendungen}: ~2.0mSv/Jahr
  \item \textbf{CT}: einmalig 10-25mSv
  \item Zigaretten, Flugreisen,\dots
  \item \textbf{Militärische Radaranlagen}: Größere Strahlenbelastung durch in Geräten erzeugte Röntgenstrahlung (\emph{nicht} durch eigentliche Radar-Mikrowellenstrahlung!)
\end{itemize}

\paragraph{Computertomographie (CT)}
\begin{itemize}
  \item[=] Computer-Auswertung vieler Röntgenaufnahmen aus versch. Richtungen
  \item \textbf{Spiralverfahren}: Patient wird mit konstanter Geschwindigkeit entlang Längsachse durch Strahlenebene bewegt, während Strahlenquelle mit konstanter Winkelgeschwindigkeit rotiert
  \item \textbf{Herkömmliches Röntgen}: Projektion von Volumen auf Fläche
  \item \textbf{CT}: 3D-Rekonstruktion aus Einzelschnitten
\end{itemize}

\paragraph{Angiographie}
\begin{itemize}
  \item[=] mit Kontrastmittel gefüllte Blutgefäße durch Röntgenstrahlung darstellen
  \item \textbf{Phlebographie}: Darstellung von arteriellen und venösen Blutgefäßen
  \item \textbf{Lymphographie}: Darstellung von Lymphgefäßen 
  \item Exakte Darst. Gefäßarchitektur \( \to \) Aufspüren von Engstellen + Blutungen
  \item \textbf{Digitale Subtraktionsangiographie} (DSA):
  \begin{enumerate}
    \item Leeraufnahme anfertigen
    \item KM über Katheter in Gefäß spritzen
    \item schnell hintereinander Aufnahmen machen
    \item Leeraufnahme subtrahieren \( \to \) Störungen (zB Knochen) ausblenden
  \end{enumerate}
  \item Informationsgewinn aus Hämodynamik (Blutbewegung)
  \item Reine Diagnose zunehmend auch mit CT/Kernspintomographie
  \item \textbf{Interventionelle Radiologie}: Angiographie zur Problembehandlung
  \item \textbf{Ballondilatation}: Aufweitung Gefäßverengungen durch winzige Ballons
  \item \textbf{Stents}: kleine Drahtkörbchen zur Gefäßwandabstützung
  \item \textbf{Coils}: kleine Platinspiralen zur inneren Blutungsverschließung
\end{itemize}

\paragraph{Magnetresonanzromographie (MRT)}
\begin{itemize}
  \item \textbf{Konzept}: Atomkerne mit ungerader Protonen-/Neutronenzahl verfügen über Eigendrehimpuls (Spin) \( \to \) werden zu winzigen Magneten
  \item Nutzung starker Magnetfelder und elektromagnetische Wechselfelder:
  \begin{enumerate}
    \item Resonante Anregung Atomkerne im Körper (meist H)
    \item Induktion elektrischer Signale in Empfängerstromkreis
  \end{enumerate}
  \item Nutzung von Wasserstoff-Spins \( \to \) besonders wasserhaltige Gewebe gut abbildbar (zB innere Organe, Rückenmark, Gehirn)
  \item Unterscheidung nach Magnetfeldstärke:
  \begin{itemize}
    \item <0.5 Tesla: Permanentmagnete oder konventionelle Elektromagnete
    \item 1.5-3 Tesla: supraleitende Magnete
  \end{itemize}
  \item \textbf{Relaxation}: Nach Abschalten von hochfrequentem Wechselfeld richten sich Spins wieder zu normalem Magnetfeld aus \\* \( \to \) unterschiedliche Abklingzeit \\* \( \to \) unterschiedliche Signalstärken (= Helligkeiten) im Bild
  \item \textbf{Gefahr}: durch magnetische Metalle am Körper
  \item \textbf{Auflösung}: klinische Standardsysteme auf ca. 1mm begrenzt
  \item \textbf{Artefakte}: Auslöschungs- und Verzerrungsartefakte
  \begin{itemize}
      \item lokale Magnetfeldinhomogenitäten
      \item Bewegungs- und Flussartefakte
      \item Funkstörungen + metallische Gegenstände
  \end{itemize}
  \item \textbf{Untersuchungsmodalitäten}
  \begin{itemize}
    \item Anatomische Bildgebung
    \item Diffusionsbildgebung (Diffusion von Wasser in Gewebe messen)
    \item Perfusionsbildgebung (Durchblutung darstellen + quantifizieren)
    \item Spektroskopie (Konzentration bestimmter Moleküle in Bereich)
    \item Funktionelle MRT (fMRT, Größen wie uB Blutfluss messen)
    \item Zelluläre Bildgebung
  \end{itemize}
\end{itemize}

\paragraph{Sonstiges}
\begin{itemize}
  \item \textbf{Koloskopie}: Darmspiegelung
  \item \textbf{Gastroskopie}: Magenspiegelung
\end{itemize}