\section*{Neurodiagnostik}

\paragraph{Diagnostik}
Unterscheidung invasive und nicht-invasive Diagnostik
\begin{itemize}
  \item \textbf{Lumbalpunktion} (Liquordiagnostik): Hohlnadel wird in Lumbalkanal auf Lendenhöhe eingeführt + Nervenwasser entnommen
  \item \textbf{Ultraschall}
  \item \textbf{Elektrophysiologie}
  \item \textbf{Positronen-Emmisions-Tomographie} (PET, Radiologische Diagnostik):
  \begin{enumerate}
    \item Zerfall eines Radionuklids \( \to \) Positron entsteht
    \item Positron trifft auf Elektron \( \to \) Annihilation
    \item Zwei Protonen entstehen (Gammastrahlung) \( \to \) Abstrahlung
    \item Winkel 180° zwischen Gammastrahlen \( \to \) treffen je auf Detektor
  \end{enumerate}
\end{itemize}

\paragraph{Elektrophysiologie}
\begin{itemize}
  \item[=] Neurophysiologie-Teilbereich, befasst sich mit elektrochemischer Signalübertragung in Nervensystem
  \item \textbf{Klinische Elektrophysiologie}: Neurologie-Teilbereich, Unterschiedliche Methoden zur Messung ganzer polysynaptischer zentraler Nervenbahnen + peripherer Nerven
  \begin{itemize}
    \item Methoden: Elektroenzephalographie (EEG), Messung evozierter Potenziale (somatosensorisch, motorisch, visuell, akustisch evoziert), Elektroneurographie (ENG) mit Messung Nervenleitgeschwindigkeit (NLG), Elektromyographie (EMG)
  \end{itemize} 
\end{itemize}

\paragraph{Elektroenzephalographie (EEG)}
\begin{itemize}
  \item Wegen CT + MRT nicht mehr häufig eingesetzt
  \item Aufzeichnung Hirnströme als Maßeinheit elektrischer Hirnaktivität (5-100 \( \mu \)V)
  \item \textbf{Oberflächen-EEG}:\@ Messung mit auf Kopfhaut aufgebrachten Elektroden
  \item Spannungsunterschiede zwischen Elektroden (= Hirnströme) auf Monitor wellenförmig sichtbar
  \item Einteilung Hirnströme nach Frequenz in 3-4 Rhythmen
  \item \textbf{Frequenz schnell} \( \to \) Person wach, Hirnaktivität normal
  \item \textbf{Frequenz langsam} \( \to \) Schlafstadium oder krankhafter Befund
  \item \textbf{Alpha-Wellen} (8-13 Hz): wach, entspannt
  \item \textbf{Beta-Wellen} (14-30 Hz): Medikamente oder fehlende Entspannung
  \item \textbf{Gamma-Wellen} (>30 Hz): Starke Konzentration
  \item \textbf{Theta-Wellen} (4-7 Hz): bei Kindern/Jugendlichen normal
  \item \textbf{Delta-/Subdelta-Wellen} (0.5-3 Hz): Tiefschlaf, Trance
  \item \textbf{Spiked/Sharp Waves}: Epilepsie
\end{itemize}

\paragraph{Nervenleitgeschwindigkeit (NLG)}
\begin{itemize}
  \item \textbf{Prinzip}:
  \begin{enumerate}
    \item Kurzer elektrischer Impuls am Arm oder Beim \( \to \) Reizung Nerv
    \item Nerv depolarisiert \( \to \) Weiterleitung in beide Richtungen
    \item[\( \to \)] Messung ausgelöste Spannungsänderung entlang Nerv
  \end{enumerate}
  \item \textbf{Berechnung}:
  \begin{itemize}
    \item Nerv an zwei Orten stimulieren
    \item Reizantworten im Muskel messen
    \item Differenz Leitungszeiten (Latenz, ms) und Reizorte (mm) bestimmen
    \item NLG = \( \Delta \text{mm} / \Delta \text{ms} \)
  \end{itemize}
  \item \textbf{Verwendung}: Ort + Schwere von Nervenschaden ermitteln, z.B. Polyneuropathie oder Nervenkompressionssyndrom (Karpaltunnelsyndrom)
\end{itemize}

\paragraph{Elektromyographie (EMG)}
\begin{itemize}
  \item[=] elektrische Muskelaktivität messen durch Einstechen von dünner Nadelelektrode in Muskel \( \to \) Ableitung von Potentialschwankungen einzelner motorischer Einheiten durch konzentrische Nadelelektroden
  \item Feststellbar, ob Muskel- oder Nerv-Erkrankung bei Muskelschwäche \\* \( \to \) Differenzierung zwischen Myo- und Neuropathien 
  \item \textbf{Intraoperativ}: EMG als Monitoring von Rückenmarksfunktion bei Wirbelsäulenoperation oder Registrierung Hirnnervenfunktionen bei Hirnstamm-Operationen
\end{itemize}

\paragraph{Evozierte Potentiale}
\begin{itemize}
  \item \textbf{Prinzip}:
  \begin{enumerate}
    \item Sinnesreiz \( \to \) el. Potentialänderung in sensorischen Großhirnrinde-Arealen
    \item Wesentlich kleinere Amplituden (schwer erfassbar)
    \item[\( \to \)] Evozierte Aktivität = \( \varnothing \) mehrere evozierte Potentiale
  \end{enumerate}
  \item \textbf{Somatosensorisch evozierte Potentiale} (SSEP): Beurteilung zentrale somatosensible Leitungsbahnen + peripherer, sensibler Nerven
  \item \textbf{Visuell evozierte Potentiale} (VEP): Beurteilung Sehnerv und -bahn
\end{itemize}

\paragraph{Sonographie (Ultraschall)}
\begin{itemize}
  \item \textbf{Ultraschall}: Schall oberhalb des hörbaren Frequenzbereichs (20 kHz - 1GHz)
  \item \textbf{Sonographie} (Echographie): Ultraschall als bildgebendes Verfahren für medizinische Untersuchung organisches Gewebe
  \item \textbf{Ultraschallgerät}: Elektronik Schallerzeugung, Signalverarbeitung + -dar\-stel\-lung, Schnittstellen für Monitor/Drucker/Speichermedien, auswechelbare Ultraschallsonde (Schallkopf)
  \item \textbf{Schallkopf}:
  \begin{itemize}
    \item Kristalle, die bei Wechselspannung mitschwingen (piezoelektrischer Effekt)
    \item Sendet Schwingungen \( \to \) unterschiedliche Reflektion d. Organe/Gewebe
    \item Impedanz: Wellen-Ausbreitung entgegenwirkender Widerstand
    \item Grauwert = Reflexionsstärke (hohe Reflexion an Grenzflächen zweier Stoffe mit großem Impedanzunterschied)
  \end{itemize}
  \item \textbf{Dopplereffekt}: Bestimmung Blutflussgeschwindigkeit \( \to \) reflektiertes Signal um Frequenz relativ zu ausgesandter verschoben (Dopplersonographie)
\end{itemize}