\section*{Gehirnfunktionen}

\paragraph{Denken und Lernen}
\begin{itemize}
  \item \textbf{Denken}: Geistige Modelle bilden + in Verbindung setzen \\* (psychologische Grundfunktion)
  \item \textbf{Lernen}: Erwerb von geistigen, körperlichen und sozialen Kenntnissen, Fähigkeiten und Fertigkeiten
\end{itemize}

\paragraph{Intelligenz}
\begin{itemize}
  \item[=] Geistige Leistungsfähigkeit; Fähigkeit, Probleme und Aufgaben effektiv + schnell lösen und in ungewohnten Situationen zurecht finden zu können
  \item \textbf{Neuropsychologie}: Neuronale Grundlagen von Intelligenz (Verarbeitung von Signalen und Informationen)
  \item \textbf{Großhirn} (Neocortex): Neurale Leistung für Intelligenz besonders relevant
  \item \textbf{Kleinhirn}, \textbf{Stammhirn} + andere phylogenetisch ältere Bereiche: für Intelligenzforschung weniger relevant
  \item \textbf{Dezentral}: Intelligenz nicht in bestimmten Gehirnbereichen lokalisiert
  \item \textbf{Generalfaktor g} vs. \textbf{multiple Intelligenzen}: Manche Forscher vermuten bereichsübergreifenden Intelligenzfaktor, andere vermuten unabhängige Intelligenzen (verbales Verständnis, räumliches Vorstellungsvermögen, \dots)
  \item \textbf{Erbe} vs. \textbf{Umwelt}: Intelligente Personen sterben mit mehr Synapsen
  \item \textbf{Intelligenzquotient}: Maß zur Bewertung intellektuelles Leistungsvermögen
  \begin{itemize}
    \item Durchschnitt 100
    \item Standardabweichung 15
    \item Frauen und Männer gleicher Mittelwert, Männer größere Varianz
  \end{itemize}
\end{itemize}

\paragraph{Medikamente}
\begin{itemize}
  \item \textbf{Methylphenidat} (Ritalin): steigert Kapazität des räuml. Arbeitsgedächtnis und Planungsfähigkeit
  \begin{itemize}
    \item Amphetamin-ähnliche Substanz, hauptsächlich bei ADHS eingesetzt
    \item Anwendung bei Narkolepsie, Steigerung Antidepressia-Wirksamkeit
    \item Vertrieben als Ritalin
  \end{itemize}
  \item \textbf{Modafinil}: steigert Leistung bei Mustererkennung + räumliches Planen, verbessert Kurzzeitgedächtnis für Zahlen
  \begin{itemize}
    \item Gehört zur Psychostimulanzen-Gruppe
    \item Behandlung bei Narkolepsie
    \item hält wach und fördert Konzentration \( \to \) \emph{brain-booster}
  \end{itemize}
  \item \textbf{Physostigmin} (Acetylcholinesterase-Hemmer): verbessert Arbeitsgedächtnis bei Gesichtserkennung
  \begin{itemize}
    \item Hydrolisiert Acetylcholin zu Essigsäure und Cholin
    \item wirkt ähnlich wie Insektizid Parathion (E 605) oder chemische Kampfstoffe Sarin und Tabun
    \item Verursacht erhöhte Acetylcholin-Konzentration in synaptischem Spalt und damit eine Erhöhung des Parasympathikotonus (Erregung)
    \item Krämpfe im Magen-Darm-Trakt, Tod durch Atemlähmung
  \end{itemize}
  \item \textbf{Erythropoetin}: steuert Bildung von Erythrozyten aus Vorgängerzellen in Knochenmark, verursacht eine Woche nach einmaliger Injektion Wortflüssigkeit (Vermutung: Erhöhung Neuroplastizität)
  \item \textbf{GTS-21}: Steigert Leistungsfähigkeit Arbeitsgedächtnis, in Zulassungsphase
\end{itemize}