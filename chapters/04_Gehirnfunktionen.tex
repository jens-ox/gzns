\section*{Gehirnfunktionen}

\paragraph{Denken und Lernen}
\begin{itemize}
  \item \textbf{Denken}: Geistige Modelle bilden + in Verbindung setzen \\* (psychologische Grundfunktion)
  \item \textbf{Lernen}: Erwerb von geistigen, körperlichen und sozialen Kenntnissen, Fähigkeiten und Fertigkeiten
\end{itemize}

\paragraph{Intelligenz}
\begin{itemize}
  \item[=] Geistige Leistungsfähigkeit; Fähigkeit, Probleme und Aufgaben effektiv + schnell lösen und in ungewohnten Situationen zurecht finden zu können
  \item \textbf{Neuropsychologie}: Neuronale Grundlagen von Intelligenz (Verarbeitung von Signalen und Informationen)
  \item \textbf{Großhirn} (Neocortex): Neurale Leistung für Intelligenz besonders relevant
  \item \textbf{Kleinhirn}, \textbf{Stammhirn} + andere phylogenetisch ältere Bereiche: für Intelligenzforschung weniger relevant
  \item \textbf{Dezentral}: Intelligenz nicht in bestimmten Gehirnbereichen lokalisiert
  \item \textbf{Generalfaktor g} vs. \textbf{multiple Intelligenzen}: Manche Forscher vermuten bereichsübergreifenden Intelligenzfaktor, andere vermuten unabhängige Intelligenzen (verbales Verständnis, räumliches Vorstellungsvermögen, \dots)
  \item \textbf{Erbe} vs. \textbf{Umwelt}: Intelligente Personen sterben mit mehr Synapsen
  \item \textbf{Intelligenzquotient}: Maß zur Bewertung intellektuelles Leistungsvermögen
  \begin{itemize}
    \item Durchschnitt 100
    \item Standardabweichung 15
    \item Frauen und Männer gleicher Mittelwert, Männer größere Varianz
  \end{itemize}
\end{itemize}

\paragraph{Medikamente}
\begin{itemize}
  \item \textbf{Methylphenidat} (Ritalin): steigert Kapazität des räuml. Arbeitsgedächtnis und Planungsfähigkeit
  \begin{itemize}
    \item Amphetamin-ähnliche Substanz, hauptsächlich bei ADHS eingesetzt
    \item Anwendung bei Narkolepsie, Steigerung Antidepressia-Wirksamkeit
    \item Vertrieben als Ritalin
  \end{itemize}
  \item \textbf{Modafinil}: steigert Leistung bei Mustererkennung + räumliches Planen, verbessert Kurzzeitgedächtnis für Zahlen
  \begin{itemize}
    \item Gehört zur Psychostimulanzen-Gruppe
    \item Behandlung bei Narkolepsie
    \item hält wach und fördert Konzentration \( \to \) \emph{brain-booster}
  \end{itemize}
  \item \textbf{Physostigmin} (Acetylcholinesterase-Hemmer): verbessert Arbeitsgedächtnis bei Gesichtserkennung
  \begin{itemize}
    \item Hydrolisiert Acetylcholin zu Essigsäure und Cholin
    \item wirkt ähnlich wie Insektizid Parathion (E 605) oder chemische Kampfstoffe Sarin und Tabun
    \item Verursacht erhöhte Acetylcholin-Konzentration in synaptischem Spalt und damit eine Erhöhung des Parasympathikotonus (Erregung)
    \item Krämpfe im Magen-Darm-Trakt, Tod durch Atemlähmung
  \end{itemize}
  \item \textbf{Erythropoetin}: steuert Bildung von Erythrozyten aus Vorgängerzellen in Knochenmark, verursacht eine Woche nach einmaliger Injektion Wortflüssigkeit (Vermutung: Erhöhung Neuroplastizität)
  \item \textbf{GTS-21}: Steigert Leistungsfähigkeit Arbeitsgedächtnis, in Zulassungsphase
\end{itemize}

\paragraph{Kognition und Gedächtnis}
\begin{itemize}
  \item \textbf{Kognition}: Oberbegriff höhere geistige Funktionen (Denken, Erkennen, Wahrnehmung, Verstand)
  \item Abgrenzung zwischen kognitiven und geistigen Fähigkeiten \\* \( \to \) Unterschied zwischen Gehirn und Geist
  \item \textbf{Gedächtnis}: Fähigkeit, Wahrnehmungen (Sinnesreize) + psychische Erlebnisse zu merken (engrammieren) + erinnern (ekphorieren)
  \begin{itemize}
    \item \textbf{Amnesie}: Gedächtnisverlust
    \item \textbf{Sensorisches Gedächtnis} (Ultrakurzzeitgedächtnis): speichert Informationen 5ms-20sec, elektrische Impulse
    \item \textbf{Arbeitsgedächtnis} (Kurzzeitgedächtnis): speichert Informationen Mi\-nu\-ten bis Tage, Bildung von Proteinen in speziellen Neuronen
    \item \textbf{Langzeitgedächtnis}: speichert Informationen über Jahre, Einlagerung der Proteine in Neuronen
  \end{itemize}
\end{itemize}

\paragraph{Gehirn --- Großhirn}
\begin{itemize}
  \item[=] \emph{cerebrum}, \emph{telencephalon}
  \item \textbf{Großhirnrinde}: äußere, Nervenzellen-reiche Sicht (graue Substanz)
  \begin{itemize}
    \item Frontallappen: motorische Funktionen
    \item Temporallappen: primärer auditorischer Cortex, Wernicke-Sprachzentrum, wichtige Gedächtnis-Strukturen (Hippocampus)
  \end{itemize}
  \item \textbf{Lateralisation}: Zuordnung zwischen körperlichen/mentalen Funktionen und Großhirnhemisphäre
  \item \textbf{Balken} (\emph{corpus callosum}): dicker Nervenstrang, verbindet beide Hemisphären
\end{itemize}

\paragraph{Gehirn --- Zwischenhirn}
\begin{itemize}
  \item Thalamus + Hypothalamus
  \item Zentren für Riech-, Seh- und Hörbahn, Oberflächensensibilität, Tiefensensibilität, emotionale Empfindung
  \item Weitere überlebenswichtige Empfindungen, Triebe und Instinkte (Hunger, Durst, Schlaf- und Fortpflanzungsbedürfnis, Überlebensinstinkt)
\end{itemize}

\paragraph{Gehirn --- Kleinhirn}
\begin{itemize}
  \item[=] Cerebellum
  \item \textbf{Kleinhirnrinde}: äußere, Nervenzellen-reiche Schicht (graue Substanz) 
  \item \textbf{Steuerung Motorik}: Koordination + Feinabstimmung, unbewusste Planung, Erlernen von Bewegungsabläufen
\end{itemize}

\paragraph{Gehirn --- Stammhirn}
\begin{itemize}
  \item[=] Mittelhirn
  \item Steuert überlebenswichtige Funktionen (Atmung, Blutdruck, Reflexe, \dots)
\end{itemize}

\paragraph{Hippocampus}
\begin{itemize}
  \item[=] Struktur, die Erinnerungen generiert
  \item Ort des Informationszusammenflusses verschiedener Sensorsysteme
  \item Verarbeitung von Informationen, Zurücksenden an Cortex
  \item Cortex speichert Gedächtnisinhalte an verschiedenen anderen Stellen
  \item wichtig für Gedächtniskonsolidierung (Überführung von Kurzzeit- zu Langzeitgedächtnis)
  \item \textbf{Anterograde Amnesie}: beide Hippocampi zerstört \( \to \) keine neuen Erinnerungen formbar, alte Erinnerungen bleiben erhalten
\end{itemize}

\paragraph{Demenz}
\begin{itemize}
  \item[=] Oberbegriff für Erkrankungsbilder mit Verlust geistiger Funktionen (Denken, Erinnern, Orientierung, Verknüpfung Denkinhalte) \\* \( \to \) alltägliche Aktivitäten nicht mehr eigenständig durchführbar
  \item \textbf{Alzheimer-Demenz}: Häufigste Demenz-Form
  \begin{itemize}
    \item Ursache: Störung im Glutamat-Gleichgewicht \\* \( \to \) Absterben von Hirnzellen \\* \( \to \) Ablagerung von Eiweis-Spaltprodukten (Amyloide) im Gehirn \\* \( \to \) Behinderung Reizübertragung \\* \( \to \) Entstehung seniler Plaques
    \item Konsequenzen: immer weniger Acetylcholin wird produziert \\* \( \to \) Glutamatkonzentration zwischen Nervenzellen durchgehend erhöht \\* \( \to \) Signale können nicht richtig erkannt/weitergeleitet werden \\* \( \to \) Nervenzelle stirbt aufgrund von Überreizung ab
    \item Behandlung: Störungen durch Antidementiva (z.B. Memantine) mindern
  \end{itemize}
  \item \textbf{Vaskuläre Demenz}: Durchblutungsstörung, plötzliche Hirnleistungs-Verschlechterung \( \to \) schlaganfallartige Symptomatik
  \item \textbf{Sekundäre Demenzen}: Verursacht durch nicht-hirnorganische Grunderkrankungen
  \begin{itemize}
    \item Rückbildung Gedächtnisstörung nach erfolgreicher Behandlung möglich
    \item Mögliche Ursachen: Stoffwechselstörungen, Schilddrüsenerkrankungen, B12-Mangel, Alkoholismus, andere chronische Vergiftungen, Infektionskrankheiten (Hirnhautentzündungen, AIDS, \dots)
  \end{itemize}
  \item \textbf{Morbus Pick}, \textbf{Fronto-Teporale Demenz} und weitere
\end{itemize}

\paragraph{Hirnhälften}
\begin{itemize}
  \item \textbf{Split-Brain-Entdeckung}: Großhirn besteht aus zwei physiologischen Hemisphären mit unterschiedlichen Funktionen
  \item Erkenntnisse:
  \begin{itemize}
    \item Stärkere Beanspruchung \emph{beider} Seiten \\* \( \leadsto \) Entwicklung einer Hirnhälfte kommt auch anderer zugute
    \item Stärkere Beanspruchung unterschiedlicher Funktionsbereiche \\* \( \leadsto \) Erhöhung Gesamtkapazität Gedächtnis
  \end{itemize}
  \item Schul- und Bildungssystem beansprucht hauptsächlich linke Seite
  \item \textbf{Gedächtnistraining}: Soll gefühl- und fantasieorientierte rechte Gehirnhälfte besser in Merkprozess einbeziehen
\end{itemize}

\paragraph{Neuroinformatik und Robotik}
\begin{itemize}
  \item \textbf{Neuroinformatik}: Informationsverarbeitung in neuronalen Systemen zur technischen Anwendung \( \to \) Arbeitsweise Gehirn simulieren
  \item \textbf{Künstliche Intelligenz}: Maschinen/Programme mit ``intelligenten'' Ergebnissen entwickeln
  \item \textbf{Computational Neuroscience}: Aus Neurobiologie, Verständnis biologisch-neuronaler Systeme durch mathematische Modelle
  \item \textbf{Robotik}:
  \begin{itemize}
    \item Stereotaktische Operationen: Platzieren von Ableitelektroden
    \item Endoskopische Operationen: Instrumente gezielt führen + exakt halten 
  \end{itemize}
\end{itemize}